\documentclass[11pt,a4paper]{article}
\usepackage[utf8]{inputenc}
\usepackage[T1]{fontenc}
\usepackage{amsmath,amsfonts,amssymb}
\usepackage{graphicx}
\usepackage{mathtools}


\title{TD2 MVFA: Linear-Time properties}
\date{}


\def\exercise#1{\Large\textbf{Exercise #1}\normalsize\\}
\def\question#1{\textbf{Question #1:}\quad}


\begin{document}
\maketitle

\exercise{1}
\def\seta{\{a\}}
\def\setab{\{a,b\}}


Let $\mathcal{TS}$ be the transition system.
$$\mathit{Traces}(\mathcal{TS})\quad=\quad\seta(\seta\setab)^\omega \quad+\quad \seta\emptyset(\setab\seta)^\omega$$

\exercise{2}
\question{1}
Let $\mathcal{TS}=(S,\mathit{Act},\rightarrow,S_0,\mathit{AP},L)$ a transition system.

We define $\mathcal{TS^*}$, another transition system without terminal states. Informally, it corresponds to $\mathcal{TS}$ without terminal states : each terminal state has been redirected to a new state which loops on itself.

$$\mathcal{TS*}=(S\uplus\{\omega\},\mathit{Act}\uplus\{\alpha\},\rightarrow^*,S_0,\mathit{AP},L)$$

In which $\omega$ is a new state, $\alpha$ a new action and $\STOP$ a new property.

We then define $\rightarrow^*$:
$$\rightarrow^*\quad=\quad\rightarrow\quad\cup\quad\{v\xrightarrow{\alpha} \omega~|~\nexists (s\in S, \beta\in\mathit{AP}), v\xrightarrow{\beta}s\}\quad\cup\quad\{\omega\xrightarrow{\alpha}\omega\}$$
Which means that every connected states in $\mathcal{TS}$ still is in $\mathcal{TS^*}$, any state with no successor is connected to $\omega$ and $\omega$ is connected to itself.


\question{2}

\exercise{3}

\exercise{4}

\exercise{5}
\question{1}
\question{2}
\question{3}
\question{4}


\end{document}
