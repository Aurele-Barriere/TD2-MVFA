\documentclass[11pt,a4paper]{article}
\usepackage[utf8]{inputenc}
\usepackage[T1]{fontenc}
\usepackage{amsmath,amsfonts,amssymb}
\usepackage{graphicx}
\usepackage{mathtools}


\title{TD2 MVFA: Linear-Time properties}
\date{}


\def\exercise#1{\Large\textbf{Exercise #1}\normalsize\\}
\def\question#1{\textbf{Question #1:}\quad}

\def\ts{\mathcal{TS}}
\def\tss{\mathcal{TS^*}}
\def\traces{\mathit{Traces}}
\def\seta{\{a\}}
\def\setab{\{a,b\}}


\begin{document}
\maketitle

\exercise{1}

Let $\ts$ be the transition system.
$$\traces(\ts)\quad=\quad\seta(\seta\setab)^\omega \quad+\quad \seta\emptyset(\setab\seta)^\omega$$

\exercise{2}

\question{1}
Let $\ts=(S,\mathit{Act},\rightarrow,S_0,\mathit{AP},L)$ a transition system.

We define $\tss$, another transition system without terminal states. Informally, it corresponds to $\ts$ without terminal states : each terminal state has been redirected to a new state which loops on itself.

$$\tss=(S\uplus\{\Omega\},\mathit{Act}\uplus\{\alpha\},\rightarrow^*,S_0,\mathit{AP},L)$$

In which $\Omega$ is a new state and $\alpha$ a new action.

We then define $\rightarrow^*$:
$$\rightarrow^*\quad=\quad\rightarrow\quad\cup\quad\{v\xrightarrow{\alpha} \Omega~|~\nexists (s\in S, \beta\in\mathit{AP}), v\xrightarrow{\beta}s\}\quad\cup\quad\{\Omega\xrightarrow{\alpha}\Omega\}$$
Which means that every connected states in $\mathcal{TS}$ still is in $\mathcal{TS^*}$, any state with no successor is connected to $\Omega$ and $\Omega$ is connected to itself.\\

\question{2} \paragraph{Lemma:} $$\traces(\tss)\ =\ \{t\in\traces(\ts)~|~t\text{ is infinite}\}\ \cup\ \{t\ \emptyset^\omega~|~t\in\traces(\ts)\text{ and $t$ is finite}\}$$
Which can be proved by double inclusion. % explain more?

\paragraph{Trace-equivalence preservation:} Let $\ts_1$ and $\ts_2$ be two transition systems such that $\traces(\ts_1)=\traces(\ts_2)$. Let $t_1\in\traces(\tss_1)$. Let us show that $t_1\in\traces(\tss_2)$.

According to the previous lemma, $t_1$ can either be in $\{t\in\traces(\ts_1)~|~t\text{ is infinite}\}$ or in $\{t\ \emptyset^\omega~|~t\in\traces(\ts_1)\text{ and $t$ is finite}\}$.

If $t_1\in\{t\in\traces(\ts_1)~|~t\text{ is infinite}\}$, then $t_1\in\traces(\ts_1)=\traces(ts_2)$. Thus, $t_1$ is an infinite trace of $\ts_2$ and then by the previous lemma, $t_1\in\traces(\tss_2)$.

If $t_1\in\{t\ \emptyset^\omega~|~t\in\traces(\ts_1)\text{ and $t$ is finite}\}$, then $t_1=t\ \emptyset^\omega$ where $t$ is a finite trace of $\ts_1$ and thus a finite trace of $\ts_2$ as well. Then by the previous lemma, $t\ \emptyset^\omega$ is also a trace of $\tss_2$.

We thus have that every trace of $\tss_1$ is a trace of $\tss_2$, which means $\traces(\tss_1)\subseteq\traces(\tss_2)$. The other inclusion is symmetric. We thus have proved that, whenever $\traces(\ts_1)=\traces(\ts_2)$, $\traces(\tss_1)=\traces(\tss_2)$.\\



\exercise{3}

\exercise{4}

\exercise{5}
\question{1}
\question{2}
\question{3}
\question{4}


\end{document}
